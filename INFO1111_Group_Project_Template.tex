\documentclass[a4paper, 11pt]{report}
\usepackage{blindtext}
\usepackage[T1]{fontenc}
\usepackage[utf8]{inputenc}
\usepackage{titlesec}
\usepackage{fancyhdr}
\usepackage{geometry}

\usepackage[english]{babel}
\usepackage{apacite}

\geometry{ margin=30mm }
\counterwithin{subsection}{section}
\renewcommand\thesection{\arabic{section}.}
\renewcommand\thesubsection{\thesection\arabic{subsection}.}
\usepackage{tocloft}
\renewcommand{\cftchapleader}{\cftdotfill{\cftdotsep}}
\renewcommand{\cftsecleader}{\cftdotfill{\cftdotsep}}
\setlength{\cftsecindent}{2.2em}
\setlength{\cftsubsecindent}{4.2em}
\setlength{\cftsecnumwidth}{2em}
\setlength{\cftsubsecnumwidth}{2.5em}


\begin{document}
\titleformat{\section}
{\normalfont\fontsize{15}{0}\bfseries}{\thesection}{1em}{}
\titlespacing{\section}{0cm}{0.5cm}{0.15cm}
\titleformat{\subsection}
{\normalfont\fontsize{13}{0}\bfseries}{\thesubsection}{0.5em}{}
\titlespacing{\section}{0cm}{0.5cm}{0.15cm}

%=======================================================================================

\begin{titlepage}
\center 
\textbf{\huge INFO1111: Computing 1A Professionalism}\\[0.75cm]
\textbf{\huge 2022 Semester 1}\\[2cm]
\textbf{\huge Practice: Team Project Report}\\[3cm]

\textbf{\huge Submission number: 01}\\[0.75cm]
\textbf{\huge Team Members:}\\[0.75cm]
\textbf{\large
    \begin{tabular}{|p{0.5\textwidth}|p{0.3\textwidth}|p{0.2\textwidth}|}
        \hline
        Name & Student ID & Levels being attempted in this submission\\
        \hline
        I don't know what's going on
        Xu Wang & 510548788 & 1, 2 \\
        Yiwei Bian & 520430468 & 1, 2 \\
        Yangyang Zhang & 520464485 & 1, 2 \\
        Yi-Yang Wang & 520508321 & 1, 2 \\
        \hline
    \end{tabular}
}\\[0.75cm]
\end{titlepage}

%=======================================================================================

\tableofcontents

%=======================================================================================

\newpage
\section*{General Instructions}

You should use this \LaTeX\ template to generate your team project report. Keep in mind the following key points:
\begin{itemize}
    \item When we assess your report, you are not given a mark. Instead we will indicate (separately, for each team member) whether each level is ''achieved''.
    \item In order to pass the unit, you must achieve at least level 1. 
    \item In order to achieve level 2, you must first have achieved level 1, and so on for each level up to level 4. This means that we will not assess a higher level until a lower level has been achieved (though we will review one level higher and give you feedback to help you in refining your work).
    \item Some parts of the report are completed as a team and other parts require each student to complete a different section. This means that for each submission, some members of the team may have completed their work for a given section, but other members may not. It also is therefore possible that some members of the team may achieve a specified level and other members of the team may not yet have achieved that level.
    \item Even if some members are completing their material for a given level, and others are not, your team members will still need to work together to edit and compile the report.  The only exception to this is where a member of the team has already achieved the level they are targeting in a previous submission and has decided to not attempt higher levels, and so is not contributing any further (this should be obvious because no level is indicated for that student on the cover page).
    \item When completing each section you should remove the explanation text and replace it with your material.
\end{itemize}

For each submission you will add new details to this report, and/or update previous sections (where previous work was not good enough to have achieved the relevant level). In particular:

\begin{itemize}
    \item \textbf{General:} For each submission, each student can attempt up to 2 levels. You must also successfully achieve each lower level before you can be assessed at a higher level. For example, in the first submission you might attempt only level 1, but not be successful in achieving that level. You then reattempt level 1 and add in level 2 in the second submission and are successful in achieving level 1 but not level 2. For the third and final submission you could then attempt level 2, or levels 2 and 3 - or even just choose to not submit anything further and remain at level 1).
    \item \textbf{Submission 1:} You should complete at least the material for level 1 (since achieving level 1 is required to pass the unit). Each member of the team can also optionally choose to complete the material for level 2.\\
    \textit{Note 1: If you do not complete the level 2 information then you obviously cannot achieve level 2 at this stage. This does not stop you from attempting level 2 in Deliverable 2 or 3, but it will make it more difficult to achieve the higher levels later in the semester.}
    \textit{Note 2: To be able to achieve Level 1 in submission one your team has to achieve level 1 in the group component (Section 1.1) and you have to achieve Level 1 in the individual component (i.e. your assigned section 1.2, 1.3, 1.4 or 1.5)}
    \item \textbf{Submission 2:} Each member of your team will complete additional sections, but because you are submitting a single document, you need to work together to compile your results together and generate the final submission.\\
    If you did not achieve level 1 in your first submission, then you should revise the material for level 1 based on the feedback, and optionally you can also complete level 2.\\
    If you achieved level 1 in your first submission, then each team member can optionally complete the material for levels 2 and 3.
    \textit{Note: If you do not achieve level 1 with this submission then the highest level you will be able to achieve in the final submission will be level 2. If you achieve level 1, but not level 2, with this submission then the highest level you will be able to achieve with the final submission is level 3.}
    \item \textbf{Submission 3:} Again, you can correct sections where you did not achieve the specified level in the previous submission, and you complete additional sections.\\
    If you still have not achieved level 1, then you should revise the material for level 1 based on the feedback, and again optionally you can also complete level 2.\\
    For those at level 1, you can choose to complete the material for levels 2 and 3.\\
    For those at level 2, you can choose to complete the material for levels 3 and 4.\\
    For those at level 3, you can choose to complete the material for level 4.
\end{itemize}

Whilst the team project is just that -- a team project -- it has been designed to also allow different members of the team to achieve different outcomes. We do expect you to work together as a team. If you do come across problems working together then the first step should be to discuss this with your tutor. Note: If you are having problems you should approach your tutor as soon as you can to make them aware of the difficulties you are having with your team.

Finally, you should also ensure that any resources you use are suitably referenced, and references are included into the reference list at the end of this document. You should use APA 6th reference style \cite{apa6}.

%=======================================================================================

\newpage
\section{Level 1: Basic Skills}

Level 1 focuses on basic technical skills (related to \LaTeX\ and Git) and the types of skills used in different computing jobs.

\subsection{Developing industry skills}

This section is completed as a team.\\
Throughout your Computing degree we will help you learn a range of new skills. Once you graduate however you will need to continue to learn new languages, new tools, new applications, etc. For this section you need to identify 5 approaches you can take to this continual learning. You should then put these in order from most effective to least effective, and then explain the circumstances in which each approach might be appropriate. (Target = $\sim$100 words per skill = $\sim$500 words total).

\subsubsection{Demand driven}

Individuals working in computer science related field are often pushed by external factors to perpetual learning due the fast integrating technologies, languages applications in the computing field. These external factors faced by an individual inevitably leads to the action of continual learning in the need to solve problems and challenges at work with a more efficient and more suitable method. Without the process of continual learning, the lack of knowledge to the newly integrated computing field could cause career stall or even replaced in the workforce, which is a common phenomenon in IT companies who have superabundant human resources in the younger generations. In addition, the insufficient knowledge of the most current technologies in the industry leads to lost shares for market leaders, which is also an undesirable outcome. To sum up, demand driven continual learning is crucial to individuals in the computing field who wants to have a sustainable and distinguished career, therefore making it the most important approach in the attempt of continual learning. 

\subsubsection{Study groups: social interactions}
Establishing social connections within the learning community can effectively help expand the learner’s knowledge base through group work and communication (Laal \& Ghodsi, 2012). Learners will have the opportunity to share their individual ideas and knowledge within the group to not only help other learners learn, but also to enhance their own understanding in the topic of study through phrasing and vocalizing their ideas. Through sharing and discussing conflicting thoughts, learners will be able to develop critical thinking  skills through approaching problems from multiple perspectives and learning to resolve problems using different methods from ones they are used to. 


\subsubsection{Feedback}
During the process of learning, individuals require effective feedback to act as guidance and motivation in learning (Vollmeyer \& Rheinberg, 2005). In both oral and written form, feedback provides individuals with an ‘assessment’ of their progress, and whether or not it is meeting their individual demand for learning outcomes. Moreover, feedback also allows individuals to gain insight into the effectiveness of their current approach to learning. If this feedback is unsatisfactory, this acts as a form of reflection and incentivises the possibility for further improvements if the current result is satisfactory. 


\subsubsection{Constructivism}
 Appropriate when learning individually, and needs to digest the knowledge based on personal construction of meaning by the learner through prior experience. Further, information can be passively received, while not being actually processed by an individual. So by making use of resources such as books, journals, online tutorial videos to develop basic prior knowledge, individuals need to construct meaning of new knowledge internally by understanding the logic behind concepts through using their own belief system. This means that individuals understand information from a distinct point of view from others, thereby learning should engage in learning experience through comparing, questioning, challenging, investigating, accepting and discarding information, and focus on personal development (Cooperstein \& Kocevar-Weidinger, 2004). 

 
\subsubsection{Conference}
The technology world is ever evolving. No matter how knowledgeable, no individual can predict or up-to-date about every technological trend. IT professionals can be well aware of the technologies they use at work, however knowing what we don’t know is a key step to continual learning. Attending conferences would give professionals opportunities to know what experts and peers do and learn from them. Technical conferences are also a good place to share technical achievements amongst professionals and a good place to obtain feedback on what may need to be improved or at the very least new directions of learning. Attending conferences would potentially solve one’s learning difficulties by putting an individual in touch with other professionals in the same field, who might have faced the same issues. Conferences could be utilized to share opinions on the subject and to absorb valuable information.



\subsection{Skills: Xu Wang : Computer Science}

This section is completed individually. Each member of the team should independently complete a separate copy of this section.\\
You should begin by allocating to each team member a different major to focus on (i.e. one of: Computer Science; Data Science; Software Development; Cyber Security). \textit{If you have a fifth member, then your tutor will suggest a fifth topic to cover}. You should then undertake research into the typical practical skills that you believe would be most important to someone who graduates with this major and is then working in industry. You should list the 8 skills that you believe are most important and for each one give a short explanation as to why you feel it is important. (Target = $\sim$100 words per skill $\sim$800 words total per student).\\
\\
Eight practical skills that are most important in computer science\\
1.\\
2.\\
3.\\
4.\\
5.\\
6.\\
7.\\
8.\\

\subsection{Skills: add student 2 name here : Data Science}

Your text goes here




\subsection{Skills: Yangyang Zhang : Software Development}
fs
\subsubsection{Professionalism}

\subsubsection{Expertise in software development methodologies}

\subsubsection{Expertise in programming language}

\subsubsection{Critical thinking}

\subsubsection{Problem solving}

\subsubsection{Mathematic skills}

<<<<<<< Updated upstream
\subsection{Skills: add student 4 name here : Cyber Security}

Your text goes here
=======



\subsection{Skills: Yiwei Bian : Cyber Security}
\subsubsection{Open mindedness / Adaptability}
The field of computer and technology is developing rapidly, and threats to systems are not guaranteed to come from old technology or methods that you are familiar with. One should always be open and informed about modern technologies and methods that may not necessarily be in frequent use but are growing rapidly so that one would not be bewildered when unfamiliar sources of threats and attacks appear. Even if a system is subject to unknown threats and attacks, it is important to be able to keep motivated to learn new knowledge in unfamiliar topics, using and learning the new knowledge efficiently to produce defence methods.  
>>>>>>> Stashed changes


%=======================================================================================

\newpage
\section{Level 2: Basic Technology}

Level 2 focuses on initial evaluation of the tech stack that is used by a selected company. All companies make use of a range of technologies, and these technologies need to work together. A tech stack is basically just this collection of technologies that collectively enable a company's systems. As an example, one of the most common technology stacks for supporting web servers is LAMP: Linux as the underlying operating system; Apache as a web server; MySQL as the supporting database; and Perl (or more recently PHP or Python) as the programming language.

Each student should choose a different tech stack and explain the role of each of the different technologies in that stack. Note that prior to researching your proposed tech stack and spending time writing about it, it might be a good idea to check with your tutor as to whether your chosen stack is suitable. (Target = $\sim$200-400 words per student).

\subsection{Tech Stack: Xu Wang}

Your text goes here

\subsection{Tech Stack: add student 2 name here}

Your text goes here

\subsection{Tech Stack: add student 3 name here}

Your text goes here

\subsection{Tech Stack: add student 4 name here}

Your text goes here


%=======================================================================================

\newpage
\section{Level 3: Advanced Skills}

Level 3 focuses on more advanced technical skills (\LaTeX\ and Git) and analysis of linkages and relationships between the items in the company tech stack.

The following is a list of advanced Git and \LaTeX\ skills/features. Each student should select one pair of items from each list and demonstrate actual use of each item (either through activity in Git, or through including items in this report). (Target = $\sim$100 words per student for each feature).
\begin{itemize}
    \item Git
    \begin{itemize}
        \item Rebasing and Ignoring files
        \item Forking and Special files
        \item Resetting and Tags
        \item Reverting and Automated merges
        \item Hooks and Tags
    \end{itemize}
    \item \LaTeX\ 
    \begin{itemize}
        \item Cross-referencing and Custom commands
        \item Footnotes/margin notes and creating new environments
        \item Floating figures and editing style sheets
        \item Graphics and advanced mathematical equations
        \item Macros and hyperlinks
    \end{itemize}
\end{itemize}

\subsection{Advanced features: Xu Wang}

Explain your use of the advanced Git and \LaTeX\ features. 

\subsection{Advanced features: add student 2 name here}

Explain your use of the advanced Git and \LaTeX\ features. 

\subsection{Advanced features: add student 3 name here}

Explain your use of the advanced Git and \LaTeX\ features. 

\subsection{Advanced features: add student 4 name here}

Explain your use of the advanced Git and \LaTeX\ features. 



%=======================================================================================

\newpage
\section{Level 4: Advanced Knowledge}

Level 4 focuses on analysing your particular tech stack and considering alternatives. Each student should consider the tech stack they described for Level 2, and then discuss each of the following points:
\begin{itemize}
    \item What are the strengths and limitations of this stack? (Target = $\sim$200 words).
    \item What alternatives exist, and under what situations might these alternatives be a better choice? (Target = $\sim$200 words).
\end{itemize}

\subsection{Advanced Knowledge: Xu Wang}

Your text goes here

\subsection{Advanced Knowledge: add student 2 name here}

Your text goes here

\subsection{Advanced Knowledge: add student 3 name here}

Your text goes here

\subsection{Advanced Knowledge: add student 4 name here}

Your text goes here



%=======================================================================================

\newpage

\bibliographystyle{apacite}
\bibliography{main}

\end{document}
\end{report}
